\chapter{Молекулярно-динамическое моделирование процессов роста и диссоциации гидрата метана}
\section{Затвердевание переохлажденной двухфазной системы метан-вода}
Нами было произведено моделирование процесса зародышеобразования и роста гидрата метана при температуре $T=210$К и давлении $p=500$ атмосфер, выполненное с различными скоростями охлаждения. Начальная конфигурация системы представляла собой 64 ($4\times 4\times 4)$ элементарные ячейки $sI$-гидрата метана, состоящая из 2944 молекул воды и 512 молекул метана. Длина ребра кубической ячейки моделирования составляла $\approx 50$ нм. В качестве используемой модели межчастичного взаимодействия была выбрана крупнозернистая модель гидрата метана, упомянутая ранее в данной работе. Информация о положениях частиц была получена из работы[ссылка], причем молекулы воды помещались в позициях атомов кислорода, а молекулы метана располагались в центрах полостей кристаллической решетки. Моделирование производилось в $NPT$-ансамбле с использованием термостата и баростата Нозе-Гувера. Интегрирование уравнений движения производилось в программном пакете моделирования классической молекулярной динамики LAMMPS[ссылка]. Временной шаг интегрирования $\tau$ был взят равным 10 фс. Применялись периодические граничные условия для всех стенок ячейки моделирования.

Причина выбора крупнозернистой модели обоснована её большей вычислительной эффективностью по сравнению со всеатомными моделями, поскольку для получения фазы гидрата из жидкой системы требуется симулировать её на протяжении длительного (вплоть до нескольких микросекунд) промежутка времени, что при использовании более сложных моделей и ограниченности имеющихся в распоряжении вычислительных ресурсов заняло бы достаточно большой (около 100 дней) срок.

На первом этапе моделирования производилось плавление кристаллической решетки гидрата метана при температуре $T=425$ К и давлении $p=100$ атмосфер в течение 20 наносекунд. За это время гидрат метана полностью расплавился, образовалась двухфазная жидкая система метан-вода, не содержащая никаких следов исходной кристаллической структуры. Затем данная конфигурация была сжата в течение 1 нс до давления $p=500$ атмосфер. После, данная конфигурация была переохлаждена в нескольких различных симуляциях до температуры $T=210$ К с высокими скоростями охлаждения $\gamma_1=10^{10}$ К/с и $\gamma_2=10^{11}$ К/с, затем наблюдался процесс затвердевания данной системы и образования гидрата метана в течение 50 нс.

Выбор именно этой температуры, соответствующей глубокому переохлаждению гидрата метана, был мотивирован тем, что при данной температуре в более ранних работах[] авторами используемой модели был достигнут быстрый, практически мгновенный рост фазы гидрата метана в системе. Кроме того, значения используемых параметров взаимодействия метан-вода $\sigma_{wm}=4,05 \si{\angstrom}$ и $\varepsilon_{wm}=0,240$ ккал/моль, несколько отличаются от приведенных нами ранее при обзоре крупнозернистой модели взаимодействия, по примеру тех же авторов. Хотя такие значения $\sigma_{wm}$ и $\varepsilon_{wm}$ дают большие значения величины растворимости метана в воде (0,0038 против 0,0022) и температуры плавления гидрата метана (301 К против 286 К) , структурные характеристики, в частности, радиальная функция распределения метан-вода, практически одинаковы в обоих случаях[ссылка].

Для каждой из скоростей охлаждения было проведено 10 независимых моделированиий, для каждого из которого была построена зависимость процентного содержания гидратов от времени моделирования.