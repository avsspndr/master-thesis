\chapter*{Введение}
\addcontentsline{toc}{chapter}{Введение}
В настоящий момент большое внимание исследователей привлекает изучение свойств газовых гидратов. Газовые гидраты представляют собой пример молекулярного комплекса «хозяин-гость».  Структура газовых гидратов представляет кристаллическую решетку, образованную молекулами воды, в полостях которой помещены молекулы газов. Молекулы воды в этом случае принято называть <<хозяевами>>, а молекулы газа-включения --- <<гостями>>. Соединения включения, имеющие подобную структуру, также принято называть клатратами. Таким образом, газовые гидраты в литературе нередко именуются клатратными гидратами. Газовые гидраты являются твердыми растворами и по структуре схожи с обычным водным льдом за тем исключением, что молекулы-гости газа обеспечивают стабильность характерной именно для газовых гидратов кристаллической решетки, составленной из пяти- и шестиугольников, объединенных в множество многогранников, соприкасающихся гранями. Такая конфигурация из молекул воды, лежащих в вершинах упомянутых многогранников распадается в отсутствии молекул-<<гостей>>. Понимание механизмов образования подобных структур представляет важную задачу в физике конденсированного состояния. 

Газовые гидраты образуются при низких температурах и высоких давлениях, в природе обнаруживаются в зонах вечной мерзлоты и на дне континентальных шельфов. Интерес к изучению данных соединений вызван перспективой их широкого практического применения в промышленности. Так, предлагается использовать газовые гидраты для транспортировки и хранения газов, поскольку их плотность в составе гидратов близка к плотности газов в сжиженном состоянии. Кроме того, способность газовых гидратов в процессе своего роста вытеснять ионы из кристаллической решетки, может использоваться для очистки воды.

Гидрат метана, являющийся наиболее распространенной формой существования клатратных гидратов природных газов, имеет большое значения для энергетической и нефтегазовой промышленностей по двум причинам. Связано это с тем, что запасы природного газа в форме гидрата, в энергетическом эквиваленте как минимум вдвое превышают запасы энергии во всех других типах углеводородоного сырья, однако еще не вполне развиты методы их извлечения. Во-вторых, давление и температура внутри труб газо- и нефтепроводов соответствуют условиям, при которых возможно возникновение гидратов, которые нарастают на стенках труб, что приводит к снижению их пропускной способности и закупориванию, соответственно неся с собой экономические потери и угрозу техногенной катастрофы. 

Природные газовые гидраты также могут рассматриваться как фактор возможного изменения климата, поскольку внезапная диссоциация гидрата метана на дне океана может привести к резкому выбросу метана в атмосферу, что вкупе с высокой парниковой активностью метана, может внести большой вклад в глобальное потепление. Учитывая вышесказанное, является важным понимание процессов роста и диссоциации гидратов и условий, что может помочь в создании ингибиторов роста гидрата метана, и в разработке методов его извлечения. Метод классической молекулярной динамики хорошо применим для исследования процессов нуклеации, поскольку этот процесс имеет пространственные и временные масштабы порядка нанометров и сотен наносекунд, характерные для упомянутого метода, в то время как существующие экспериментальные наблюдения зародышеообразования затруднены, чем и объясняется его популярность в данной области.

Цели работы:
\begin{enumerate}
\item Исследование структурных характеристик гидрата метана, образующегося из жидкой переохлажденной двухфазной системы метан-вода;
\item Исследование особенностей процесса диссоциации гидрата метана, подвергнутого нагреву с постоянной скоростью.
\end{enumerate}

Глава 1 посвящена проблематике исследования газовых гидратов. В главе 2 произведен обзор метода классической молекулярной динамики, используемых моделей гидрата метана и алгоритма идентификации структуры. В главе 3 нами представлены результаты молекулярно-динамического моделирования системы метан-вода с использованием крупнозернистой модели, подвергнутой охлаждению с различными скоростями охлаждения, в область фазовой диаграммы, соответствующей возникновению гидрата метана. Также в этой главе приведены результаты моделирования диссоциации гидрата метана при нагревании образца под высоким давлением в рамках всеатомной модели.