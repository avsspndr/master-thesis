\chapter*{Введение}
\addcontentsline{toc}{chapter}{Введение}
В настоящий момент большое внимание исследователей привлекает изучение свойств газовых гидратов. Газовые гидраты представляют собой твердые растворы газов в воде, кристаллическая структура которых образована совокупностью соприкасающихся полостей-многогранников, в вершинах которых находятся молекулы воды, содержащих внутри одну или несколько молекул газа.

Газовые гидраты образуются при низких температурах и высоких давлениях и в природе обнаруживаются в зонах вечной мерзлоты и на дне континентальных шельфов. Интерес к изучению данных соединений вызван перспективой их широкого практического применения в промышленности[ссылка]. Так, предлагается использовать газовые гидраты для транспортировки и хранения газов, поскольку их плотность в составе гидратов близка к плотности газов в сжиженном состоянии. Кроме того, способность газовых гидратов в процессе своего роста вытеснять ионы из кристаллической решетки, может использоваться для очистки воды.

Гидраты природных газов, в особенности, наиболее распространенный гидрат метана, играют большую роль в нефтегазовой промышленности по двум причинам. Во-первых, по имеющимся оценкам, запасы природного газа в форме гидрата, в энергетическом эквиваленте намного превышают запасы всех других типов углеводородов, однако еще не вполне развиты методы их извлечения. Во-вторых, термобарические условия на стенках газо- и нефтепроводов соответствуют условиям образования гидратов природных газов, которые нарастают на стенках труб, что ведет к снижению их пропускной способности и закупориванию. 

Природные газовые гидраты также могут рассматриваться как фактор возможного изменения климата, поскольку внезапная диссоциация гидрата метана на дне океана может привести к резкому выбросу метана в атмосферу, что вкупе с высокой парниковой активностью метана, может внести большой вклад в глобальное потепление. Учитывая вышесказанное, является важным понимание процессов нуклеации, роста и диссоциации гидратов и условий, при которых они протекают.


Метод классической молекулярной динамики хорошо применим для исследования процессов нуклеации, поскольку этот процесс имеет пространственные и временные масштабы порядка нанометров и сотен наносекунд, характерные для упомянутого метода. В настоящей работе представлены результаты молекулярно-динамического моделирования системы метан-вода с использованием крупнозернистой модели, подвергнутой охлаждению с различными скоростями охлаждения, в область фазовой диаграммы, соответствующей возникновению гидрата метана. Также были исследованы процессы диссоциации гидрата метана при нагревании образца под высоким давлением в рамках всеатомной модели.