\chapter{Гидрат метана}
\section{Краткий обзор}
\par Гидрат метана относится к классу веществ, называемых газовыми гидратами. Структура газовых гидратов представляет кристаллическую решетку, образованную молекулами воды, в полостях которой помещены молекулы газов. Молекулы воды в этом случае принято называть <<хозяевами>>, а молекулы газа-включения --- <<гостями>>. Соединения включения, имеющие подобную структуру, также принято называть клатратами. Таким образом, газовые гидраты в литературе нередко именуются клатратными гидратами. Газовые гидраты являются твердыми растворами и по структуре схожи с обычным водным льдом за тем исключением, что молекулы-гости газа обеспечивают стабильность характерной именно для газовых гидратов кристаллической решетки, составленной из пяти- и шестиугольников, объединенных в множество многогранников, соприкасающихся гранями. Такая конфигурация из молекул воды, лежащих в вершинах упомянутых многогранников распадается в отсутствии молекул-<<гостей>>.
\par Газовые гидраты впервые были открыты в 1811 году британским химиком Дэвидом Гемфри, который обнаружил[ссылка], что водный раствор хлора кристаллизуется более охотно, чем обычная вода и чистый хлор, который не претерпевает никаких изменений при охлаждении до температуры -40$^{\circ}$.