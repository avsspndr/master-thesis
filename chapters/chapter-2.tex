\chapter{Гидрат метана}
\section{Историческая справка}
\par Гидрат метана относится к классу веществ, называемых газовыми гидратами. Структура газовых гидратов представляет кристаллическую решетку, образованную молекулами воды, в полостях которой помещены молекулы газов. Молекулы воды в этом случае принято называть <<хозяевами>>, а молекулы газа-включения --- <<гостями>>. Соединения включения, имеющие подобную структуру, также принято называть клатратами. Таким образом, газовые гидраты в литературе нередко именуются клатратными гидратами. Газовые гидраты являются твердыми растворами и по структуре схожи с обычным водным льдом за тем исключением, что молекулы-гости газа обеспечивают стабильность характерной именно для газовых гидратов кристаллической решетки, составленной из пяти- и шестиугольников, объединенных в множество многогранников, соприкасающихся гранями. Такая конфигурация из молекул воды, лежащих в вершинах упомянутых многогранников распадается в отсутствии молекул-<<гостей>>.
\par Газовые гидраты впервые были открыты в 1811 году британским химиком Дэвидом Гемфри, который обнаружил[ссылка], что водный раствор хлора кристаллизуется более охотно, чем обычная вода и чистый хлор, который не претерпевает никаких изменений при охлаждении до температуры -40$^{\circ}$. В течение 125 лет с момента данного открытия исследователи в основном занимались поиском как можно большего числа соединений, способных образовывать гидраты, а также описанием их состава и физических свойств. Так, в 1823 году Фарадей предположил химический состав гидрата: \ce{Cl * 10H2O}, что было экспериментально подтверждено Розебомом в 1884 г. В 1888 году Виллард получил гидраты метана, этана и пропана, а также выдвинул гипотезу, что гидраты являются кристаллами. В 1902 году Форкран определил равновесные температуры при атмосферном давлении для 15 различных гидратов.

\par В 1934 году Хаммершмидт обнаружил, что природные газы и вода, в небольшом количестве содержащаяся в объеме газопроводов, образуют гидраты, впоследствии закупоривая их. В связи с этим обстоятельством тематика изучения газовых гидратов стала заметно интереснее и с практической точки зрения, были начаты исследования их термодинамических и кинетических свойств. Так, вскоре был введен строгий контроль за содержанием \ce{H2O} в трубопроводах, а в 1930-1950 годах ученые вели поиск различных ингибиторов, подавляющих рост гидратов, таких как соли хлора,метанол и моноэтиленгликоль.

\par Штакельберг, Мюллер и Клауссен в экспериментах по ренгтеновской дифракции идентифицировали кристаллическую структуру различных газовых гидратов и выделили два её типа: \textit{sI} и \textit{sII}. В 1987 г. группой Рипмистра была обнаружена гексагональная структура гидрата \textit{sH}.

\par В 1959 Ван-дер-Ваальс и Платье разработали  феноменологическую статистическую модель для оценки термодинамических свойств газовых гидратов, которая является наиболее популярной и в настоящее время. Данная модель позволяет предсказывать макроскопические характеристики, такие как температура и давление на основе межмолекулярных потенциалов взаимодействия. Достоинством теории Ван-дер-Ваальса-Платье помимо точности предсказаний является возможность расчета свойств гидратов смесей газов, основываясь на характеристиках чистых газов гидратообразователей.

\par В 1965 году были найдены природные залежи гидратов природных газов на мессояхском газовом месторождении, после чего было обнаружено множество других залежей гидратов на дне океана и в зонах вечной мерзлоты. Возникло понимание, что газовые гидраты могут быть потенциальным источником углеводородной энергии в будущем.

