\chapter{Гидрат метана}
\section{Историческая справка}
\par Гидрат метана относится к классу веществ, называемых газовыми гидратами. Структура газовых гидратов представляет кристаллическую решетку, образованную молекулами воды, в полостях которой помещены молекулы газов. Молекулы воды в этом случае принято называть <<хозяевами>>, а молекулы газа-включения --- <<гостями>>. Соединения включения, имеющие подобную структуру, также принято называть клатратами. Таким образом, газовые гидраты в литературе нередко именуются клатратными гидратами. Газовые гидраты являются твердыми растворами и по структуре схожи с обычным водным льдом за тем исключением, что молекулы-гости газа обеспечивают стабильность характерной именно для газовых гидратов кристаллической решетки, составленной из пяти- и шестиугольников, объединенных в множество многогранников, соприкасающихся гранями. Такая конфигурация из молекул воды, лежащих в вершинах упомянутых многогранников распадается в отсутствии молекул-<<гостей>>.
\par Газовые гидраты впервые были открыты в 1811 году британским химиком Дэвидом Гемфри, который обнаружил[ссылка], что водный раствор хлора кристаллизуется более охотно, чем обычная вода и чистый хлор, который не претерпевает никаких изменений при охлаждении до температуры -40$^{\circ}$. В течение 125 лет с момента данного открытия исследователи в основном занимались поиском как можно большего числа соединений, способных образовывать гидраты, а также описанием их состава и физических свойств. Так, в 1823 году Фарадей предположил химический состав гидрата: \ce{Cl * 10H2O}, что было экспериментально подтверждено Розебомом в 1884 г. В 1888 году Виллард получил гидраты метана, этана и пропана, а также выдвинул гипотезу, что гидраты являются кристаллами. В 1902 году Форкран определил равновесные температуры при атмосферном давлении для 15 различных гидратов.

\par В 1934 году Хаммершмидт обнаружил, что природные газы и вода, в небольшом количестве содержащаяся в объеме газопроводов, образуют гидраты, впоследствии закупоривая их. В связи с этим обстоятельством тематика изучения газовых гидратов стала заметно интереснее и с практической точки зрения, были начаты исследования их термодинамических и кинетических свойств. Так, вскоре был введен строгий контроль за содержанием \ce{H2O} в трубопроводах, а в 1930-1950 годах ученые вели поиск различных ингибиторов, подавляющих рост гидратов, таких как соли хлора,метанол и моноэтиленгликоль.

\par Штакельберг, Мюллер и Клауссен в экспериментах по ренгтеновской дифракции идентифицировали кристаллическую структуру различных газовых гидратов и выделили два её типа: \textit{sI} и \textit{sII}. В 1987 г. группой Рипмистра была обнаружена гексагональная структура гидрата \textit{sH}. Данные типы структур являются наиболее распространенными кристаллическими модификациями газовых гидратов. Сообщалось так же о существовании гораздо реже встречающихся фазах, возникающих сверхвысоких давлениях порядка 1 ГПа, например, Дядиным [ссылка].

\par В 1959 году Ван-дер-Ваальс и Платье разработали  феноменологическую статистическую модель для оценки термодинамических свойств газовых гидратов, которая является наиболее популярной и в настоящее время. Данная модель позволяет предсказывать макроскопические характеристики, такие как температура и давление на основе межмолекулярных потенциалов взаимодействия. Достоинством теории Ван-дер-Ваальса-Платье помимо точности предсказаний является возможность расчета свойств гидратов смесей газов, основываясь на характеристиках чистых газов гидратообразователей.


\par В 1965 году были найдены природные залежи гидратов природных газов на мессояхском газовом месторождении, после чего было обнаружено множество других залежей гидратов на дне океана и в зонах вечной мерзлоты. Возникло понимание, что газовые гидраты могут быть потенциальным источником углеводородной энергии в будущем.
    
\par В 1997 году было обнаружено, что полости гидратов могут содержать не обязательно одну, а целых две молекулы-гостя, что было продемонстрировано примере sII гидрата, который в своей большей полости содержал 2 молекулы азота.

\section{Кристаллическая структура}

\par Кристаллическая структура клатратных гидратов природных газов в основном представлена в трех наиболее распространенных формах: \textit{sI}, \textit{sII} и \textit{sH}. Перед тем как обсудить конкретное строение указанных структур, необходимо отметить присущие им общие черты.
\par Молекулы воды, связанные водородными связами, образуют так называемые <<полости>> --- пространственные области в форме многранников, грани которых представляют собой четырехугольники, пятиугольники и шестиугольники. Молекулы воды при этом находятся в вершинах указанных многугольников, а водородные связи направлены вдоль их рёбер. Различные полости соединяются друг с другом, разделяя грани между собой. Получившийся <<ажурный>> каркас носит название решетки-<<хозяина>> Молекулы газов-включений располагаются внутри полостей и называются <<гостями>>. Упомянутые многогранники могут содержать различное число многоугольников в зависимости от типа кристаллической решетки. Обознать их принято используя запись типа $n^m$, где $m$ -- количество многоугольников, составленных из $n$ вершин. В этом случае, например, запись $5^{12}6^4$ будет соответстовать многограннику, построенному из 12 пятиугольников и 4 шестиугольников.
\par Кристаллическая структура \textit{sI} представлена полостями $5^{12}$ и $5^{12}6^2$, структура \textit{sII} полостями $5^12$ и $5^{12}6^4$, а \textit{sH} -- полостями $5^{12}$, $5^{12}6^8$ и $4^35^66^3$. Элементарные ячейки данных кристаллических структур приведены на рис. \ref{fig1.2.1}. На каждую элементарную ячейку приходится соответственно 46, 136 и 34 молекулы воды.

\par Существование нескольких структур объясняется тем фактом, что газовые гидраты образуются молекулами-гостями с различными размерами. Так, \textit{sI}-гидраты образованы молекулами с диаметром от 4,2 до 6 \si{\angstrom}. Примерами являются газы \ce{CH4, C2H6, CO2, H2S}. Молекулы размером меньше 4,2 \si{\angstrom}, такие как водород, азот, криптон, кислород, аргон, а так же молекулы, например, изобутана или пропана диаметром 6-7 \si{\angstrom}, формируют \textit{sII}-гидраты. Наконец, наиболее крупные молекулы размером 7-9 \si{\angstrom}, в совокупности с молекулами меньшего размера, образуют \textit{sH}-гидраты.

\begin{figure}[H]
    \centering
    \includegraphics[width=.9\linewidth]{figures/hydrstruct.png}
    \caption{Кристаллическая структура трех наиболее распространенных типов гидратов, а также 5 различных полостей, образующих их. Источник[Gas hydrates in sustainable chemistry / A. Hassanpouryouzband [et al.] // Chem. Soc. Rev. – 2020. – Vol. 49. – P. 5225]}
    \label{fig1.2.1}
\end{figure}

\par Газовые гидраты состоят из воды на 85\% и по этой причине их часто сравнивают с обычным льдом. Однако в отличие от кристаллической решетки льда, характерная для гидратов структура не может существовать в отсутствие молекул-гостей. Кроме того, гидраты отличаются большей механической прочностью по сравнению со льдом[http://refhub.elsevier.com/S1875-5100(18)30492-X/sref46], обладают меньшей теплопроводностью[https://pubs.acs.org/doi/abs/10.1021/j100282a026] и большей теплоемкостью по сравнению с гексагональным льдом.

\par Важно отметить, что газовые гидраты являются нестехиометричными соединениями и идеальных кристаллов газовых гидратов не существует, поскольку молекулы воды всегда присутствуют в большем  количестве чем молекулы газов. Поэтому еще одной характеристикой гидратов является число заполнения полостей. Идеальное стехиометрическое отношение для структуры \textit{sI} составляет $1:5^{3/4}$, \textit{sII} и \textit{sH} -- $1:5^{2/3}$ [1-s2.0-S187551001830492X-main (2)].

\pagebreak
\section{Образование гидратов}
\par Процесс формирования гидратов принято разделять на стадию нуклеации и стадию устойчивого роста кристаллической фазы. Образование газовых гидратов из растворенных в воде газов возможно только в условиях низких температур и высоких давлений. В случае неполярных молекул-гостей, например, метана причина этого заключается в том, что \ce{CH4} является гидрофобной молекулой и имеет низкую величину растворимости в воде порядка $10^{-3}-10^{-5}$ мольных долей при обычных условиях. Однако при понижении температуры до отрицательных величин и увеличения давления до десятков МПа, растворимость  \ce{CH4} увеличивается на два порядка.

\par При исследовании клатратных гидратов под нуклеацией обычно подразумевают явление гомогенной нуклеации, которое представляет собой возникновение в исходной фазе чистого, то есть без наличия примесных молекул или атомов, вещества зародыша новой кристаллической или аморфной фазы, который может состоять из десятков или тысяч частиц. В объеме переохлажденной жидкости вследствие флуктуаций периодически образуются и распадаются кластеры (зародыши) из молекул воды, выстраивающихся вокруг растворенного метана.

\par Зародышеообразование является стохастическим явлением: с некоторой вероятностью зародыш может достичь некоторого критического размера $r$, после чего происходит устойчивый рост новой твердой фазы. Существование критического размера можно объяснить если обратить внимание на величину избытка свободной энергии Гиббса $\Delta G$, который равен сумме поверхностной и объемной частей свободных энергий зародыша:

\begin{equation}
\Delta G = \Delta G_{surf} + \Delta G_{vol} = 4\pi r^2 \sigma + \frac{3}{4}\pi r^3 \Delta g_{vol} \, .
\label{eq1.3.1}
\end{equation}

Здесь $\sigma$ -- поверхностное натяжение на границе раздела зародыша новой фазы и раствора, $\Delta g_{vol}$ изменение свободной энергии в пересчете на единицу объема. Как видно из рисунка \ref{fig1.3.1} $\Delta G$ имеет максимум, соответствующий критическому значению радиуса, который равен

\begin{equation}
    r_{crit} = -2\sigma/\Delta g_{vol} \,.
    \label{eq1.3.2}
\end{equation}
Соотношение \ref{eq1.3.2} можно получить, взяв производную \ref{eq1.3.1} по радиусу зародыша и приравняв её нулю. Соответствующий максимум избытка свободной энергии имеет вид
\begin{equation}
    \Delta G_{crit} = 4\pi\sigma r_{crit}^2 / 3 \,.
    \label{eq1.3.3}
\end{equation}

\begin{figure}[H] 
    \centering
    \includegraphics[width=.9\linewidth]{figures/hydrnucl.png}
    \caption{Диаграмма, иллюстрирующая рост гидрата метана. Источник[https://doi.org/10.1038/s42004-021-00539-]}
    \label{fig1.3.1}
\end{figure}

Частота с которой возникают зародыши критического размера в значительной степени зависит от высоты энергетического барьера. При уменьшении величины $r_{crit}$ величина $\Delta G$ так же падает. Критический радиус зародыша может уменьшаться в зависимости от величины пересыщения и переохлаждения $\Delta T$ раствора по зависимости $r_{crit} \propto 1/\Delta T$[http://refhub.elsevier.com/S1875-5100(18)30492-X/sref125]. Соответственно увеличивается и вероятность зародышеообразования. 

В настоящее время считается, что образованию зародыша критического размера предшествует возникновение так называемого \textit{сгустка} (с англ. \textit{blob}) -- аморфного кластера, состоящего из нескольких молекул газа, разделенных между собой молекулами воды, возникающего вследствие локальных флуктуаций концентрации гостей. Концентрация молекул газов в сгустке выше чем, во всем остальном растворе, с которым он находится в динамическом равновесии. В пределах сгустка периодически возникают и исчезают гидратные полости и их фрагменты, адсорбируя окружающий газ и вновь концентрируя его. Впоследствии может возникнуть кластер критического размера. Критический зародыш может быть как аморфным, так и кристаллическим, в зависимости от того, какие типы полостей образуются первоначально. Соответственно возникает либо аморфная фаза гидрата, которая потом преобразуется в кристаллическую сама по себе со временем или путем отжига, либо же кристалл образуется напрямую.

\section{Метод классической молекулярной динамики}
\par Согласно представлениям классической физики, полностью описать движение произвольной механической системы, состоящей из $N$ частиц можно, если знать координаты $\mathbf{r}_i(t)$ и скорости $\mathbf{\dot{r}}_i(t)$ всех частиц системы $i=1...N$ в любой момент времени $t$. В таком случае движение каждой частицы $i$, находящейся в силовом поле остальных, описывается классическими уравнениями Ньютона:

\begin{equation}
    m_i \mathbf{\ddot{r}}_i(t) = \mathbf{F}_i(t) = \sum\limits_{j=1}^{N} \mathbf{F}_{ij}(t)\,,
    \label{eq1.4.1}
\end{equation}
где $m_i$ -- масса частицы, $\mathbf{\ddot{r}}_i(t)$ -- её ускорение, $\mathbf{F}_i(t)$ -- результирующая сила, действующая на неё, $\mathbf{F}_{ij}$ -- сила действующая на $i$-ю частицу со стороны $j$-й частицы. Если известна потенциальная энергия парного взаимодействия от расстояния $r_{ij}$ между ними $U(|\mathbf{r}_i-\mathbf{r}_j|) = U(r_{ij})$, то сила выражается следующим образом:

\begin{equation}
    \mathbf{F}_{ij} = - \dfrac{\partial U(r_{ij})}{\partial r_{ij}}\cdot\dfrac{\mathbf{r}_{ij}}{r_{ij}}\,.
    \label{eq1.4.2}
\end{equation}
Подставляя \eqref{eq1.4.2} в \eqref{eq1.4.1} получим дифференциальное уравнение

\begin{equation}
    \mathbf{\ddot{r}} = -\dfrac{1}{m_i}\sum\limits_{j\neq i}^N \dfrac{\partial U(r_{ij})}{\partial r_{ij}}\cdot\dfrac{\mathbf{r}_{ij}}{r_{ij}}\,.
    \label{eq1.4.3}
\end{equation}
Данное векторное уравнение распадается на 3 скалярных. Учитывая, что для полного описания системы нужно решить его для всех $N$ частиц, получаем систему из $6N$ дифференциальных уравнений. Также необходимо задать положения частиц и их скорости в начальный момент времени $t_0$:

\begin{equation}
    \begin{cases}
        \mathbf{r}_{i}(t_0) = \mathbf{r}_0\, , \\
        \mathbf{\dot{r}}_{i}(t_0) = \mathbf{\dot{r}}_0\, , \\
        i = 1...N\,.
        \label{eq1.4.4}
    \end{cases}
\end{equation}

\par Молекулярная физика исходит из представления о том, что любой материал состоит из очень большого числа атомов или молекул, взаимодействующих между собой и находящихся в непрерывном движении. Предполагая, что взаимодействие между ними является классическим, что, как можно показать, справедливо для достаточного широкого класса веществ среди которых встречаются газы, жидкости, кристаллы, аморфные тела, можно применить приведенные выше уравнения для описания поведения этих систем.

\par При этом как известно, состояние макроскопической системы в термодинамически равновесном состоянии определяется сравнительно небольшим набором параметров, такими как температура, давление, плотность, концентрация, химический потенциал, энтропия, внутренняя энергия и другие. В то же время, если учитывать что макроскопические объемы вещества содержат порядка $10^{23}$ атомов или молекул и описываются соответствующим числом степеней свободы, отсюда следует существование огромного числа микросостояний. Поэтому одному макроскопическому термодинамическому состоянию может соответствовать целая совокупность микросостояний, реализующих его, называемая \textit{статистическим ансамблем}. Оперируя методами статистической механики, можно установить связь между макроскопическими и микроскопическими величинами, характеризующими исследуемую систему.

\par Аналитическое решение уравнений Ньютона для системы, состоящей из столь большого числа атомов не представляется возможной, однако с развитием вычислительных методов и ростом мощностей ЭВМ появилась возможность численного решения рассматриваемых уравнений.