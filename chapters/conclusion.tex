\chapter*{Заключение}
\addcontentsline{toc}{chapter}{Заключение}
В данной работе произведено моделирование процесса образования гидрата метана из переохлажденной жидкой системы метан-вода. Были выявлены особенности гидратообразования в зависимости от различной скорости охлаждения до исходной температуры, при которой рассматривалось затвердевание. Так, при большей скорости охлаждения, возникновению критического зародыша в среднем соответствует более ранний момент времени, рост гидрата метана происходят быстрее. Более детальный сравнительный анализ полученных структур возможно было бы осуществить с использованием алгоритмов, позволяющих идентифицировать отдельные типы полостей гидратов.

Диссоциация $sI$-гидрата метана, смоделированная для восьми независимых систем, во всех случаях имеет одинаковую динамику и не демонстрирует стохастичности, как в случае зародышеобразования. Плавление кристаллической решетки происходит при температуре 425 К для времени моделирования 10 нс, что соответствует скорости отжига $2\cdot 10^{10}$ К/с. В связи с последним обстоятельством возникает интерес к поиску ответов на следующие вопросы: каким образом зависит температура плавления идеального кристалла гидрата метана в зависимости от скорости отжига? Как вид этой зависимости различается для решеток $sI$ и $sII$ типов или для различных <<гостей>>?